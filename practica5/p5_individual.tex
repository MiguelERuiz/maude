\documentclass[a4paper,11pt]{report}

%%%%%%%%%%%%%%%%%%%%%%%%%%%%%%%%%%%%%%%%%%%%%%%%%%%%%%%%%%%%%%%%%%%%%%%
% Definicion de paquetes
\usepackage[T1]{fontenc}
\usepackage[utf8]{inputenc}
\usepackage[spanish]{babel}
\usepackage[printsolution=true]{exercises}
\usepackage[doublespacing]{setspace}% just to set the samples further apart
\usepackage[colorinlistoftodos,prependcaption,textsize=tiny]{todonotes}
%%%%%%%%%%%%%%%%%%%%%%%%%%%%%%%%%%%%%%%%%%%%%%%%%%%%%%%%%%%%%%%%%%%%%%%
% Definición de comandos
\setlength{\marginparwidth}{2cm}
\newcommand{\unsure}[2]{\todo[linecolor=red,backgroundcolor=red!25,bordercolor=red,#1]{#2}}
\newcommand{\change}[2]{\todo[linecolor=blue,backgroundcolor=blue!25,bordercolor=blue,#1]{#2}}
\newcommand{\info}[2]{\todo[linecolor=OliveGreen,backgroundcolor=OliveGreen!25,bordercolor=OliveGreen,#1]{#2}}
\newcommand{\improvement}[2]{\todo[linecolor=Plum,backgroundcolor=Plum!25,bordercolor=Plum,#1]{#2}}
\newcommand{\thiswillnotshow}[2]{\todo[disable,#1]{#2}}
\newcommand*{\SignatureAndDate}[1]{%
    \par\noindent\makebox[2.5in]{\hrulefill} %\hfill\makebox[2.0in]{\hrulefill}%
    \par\noindent\makebox[2.5in][l]{#1}      %\hfill\makebox[2.0in][l]{Date}%
}%
%%%%%%%%%%%%%%%%%%%%%%%%%%%%%%%%%%%%%%%%%%%%%%%%%%%%%%%%%%%%%%%%%%%%%%%
%% Empieza el documento
\begin{document}
	\title{Práctica 5 - Individual}
	\author{
		Miguel Emilio Ruiz Nieto
	}

	\maketitle

  \section*{Ejercicio 3}

  % Hablar sobre el resultado de la última propiedad, que da un contraejemplo
  La propiedad ``Nunca nadie con entrada va a los recreativos'' no se satisface
  debido a que da un contraejemplo. El motivo es que sí que es posible ir a los
  recreativos con entrada ya que se puede comprar la entrada, ponerse en la
  cola y no entrar porque o bien el aforo esté completo o bien porque no seas
  mayor de edad, y te echen a la Plaza Mayor. Una vez allí, el sistema permite
  ir a los recreativos con la entrada comprada, por tanto, el hecho de que la
  propiedad no se cumpla es un comportamiento correcto, como se muestra en la
  siguiente figura:
  \begin{figure}[h]
    \includegraphics[scale=0.5]{lugares.png}
    \centering
    \caption{Flujo de las personas genéricas en el sistema}
  \end{figure}

  \section*{Ejercicio 4}

  % Hablar sobre las reglas que no pueden convertirse en ecuaciones
  % indicando los motivos, así como las que sí se pueden
\end{document}